\documentclass[11pt,a4paper,twoside]{book}
\usepackage[lmargin=3cm,rmargin=2.5cm,tmargin=2.5cm,bmargin=2.5cm]{geometry} %inside margin is bigger than outside margin
\usepackage{graphicx} %for figures

\usepackage[francais]{babel} %comments if you are writing your thesis in english

\usepackage{minitoc} %to have mini table of contents/sommaire at the beginning of each chapter

\usepackage{lipsum} 

\usepackage{tikz} %to create a frame in the résumé/abstract in the fouth cover

%-- To create blank pages --%
\usepackage{afterpage}
\makeatletter %@ as a letter, to use it in the following commands
\newcommand*\cleartoleftpage{ % macro defining a command to force a page to be on the right side (e.g for resume/abstract page in particular)
	\clearpage
	\ifodd\value{page}\hbox{}\newpage\fi
}
\renewcommand*{\cleardoublepage}{\clearpage\if@twoside \ifodd\c@page\else
	\hbox{}%
	\thispagestyle{empty}%
	\newpage%
	\if@twocolumn\hbox{}\newpage\fi\fi\fi} %to put the following page on the right and add a blank page if necessary
\makeatother %makes @ as a symbol/number again (comes with \makeatletter in the preamble)
%------------------------------%

%-- Defines headers and footers of each page --%
\usepackage{fancyhdr} 
\pagestyle{fancy}
\fancyhf{} %clears the headers
\fancyhead[LE,RO]{\nouppercase\rightmark} %section name
\fancyhead[RE,LO]{\leftmark}              %chapter name
\fancyfoot[CE,CO]{\thepage}
%header alternates from odd page to even page
%------------------------------%


%-- Commands to make the title page --% %do not delete, except if you do not want to use the suggested title page
%\usepackage{fullpage}
\makeatletter %@ as a letter, to use it in the following commands
\DeclareRobustCommand\subtitle[1]{\gdef\@subtitle{#1}}
\DeclareRobustCommand\lab[1]{\gdef\@lab{#1}}
\DeclareRobustCommand\rfield[1]{\gdef\@rfield{#1}}
\DeclareRobustCommand\supervisors[1]{\gdef\@supervisors{#1}}
\DeclareRobustCommand\defdate[1]{\gdef\@defdate{#1}}
\DeclareRobustCommand\docschool[1]{\gdef\@docschool{#1}}
\DeclareRobustCommand\logoLab[1]{\gdef\@logoLab{#1}}
%------------------------------%

%-- 1. TITLE PAGE FORMATING --%
\logoLab{defaultLogoLab} %fill with the name of the logo of your lab
\docschool{[Ecole doctorale]}
\author{[Prenom NOM]}
\title{\LARGE{Ending the HIV epidemic thanks to voluntary testing? A game theoretical approach.}}
\title{[Ecrire ici le titre de la thèse en français.
	 Voici un titre qui est long car il est sur plusieurs lignes. Ajoutons encore une ligne.]}
\subtitle{[sous titre un peu long, sous titre un peu long  sous titre un peu long  sous titre un peu long  sous titre un peu long]}
\lab{[Laboratoire / Equipe de recherche]}
\rfield{[Discipline]} % Rsearch field, e.g. biomathematics
\supervisors{Prenom(s) NOM(S)} %Prénom et Nom du (ou des) Directeur(s) de thèse
\date{\today}
\defdate{[date de soutenance]}
\newcommand{\jury}{
	\begin{tabular}{lclcl}
		Prenom NOM &~& Maitre de conférence &~& Examinateur.ice\\ 
		Prenom NOM &~& Fonction &~& Role\\
		Prenom NOM &~& Fonction &~& Role\\
		Prenom NOM &~& Fonction &~& Role\\
		Prenom NOM &~& Fonction &~& Role\\
		Prenom NOM &~& Fonction &~& Directeur.ice\\
	\end{tabular}	
}
%------------------------------%


\begin{document}
	
	%%% Makes the title page %%%
	\begin{titlepage}	
	\newgeometry{left=3cm,right=3cm,bottom=0.1cm}
	\begin{center}
		\begin{figure}
			\centering
			\vspace*{-1.5cm}
			\begin{minipage}{.38\textwidth}
				\centering
				\includegraphics[height=2.3cm]{logos/logoSU}
			\end{minipage} \hfill
			\begin{minipage}{.38\textwidth}
				\centering
				\includegraphics[height=2.3cm]{logos/\@logoLab}
			\end{minipage}
		\end{figure}
		\vspace*{0.5cm}
		{\Huge Sorbonne Université}\\
		\vspace*{0.8cm}
		{\LARGE \@docschool}\\
		\vspace*{0.5cm}
		{\LARGE\textit{ \@lab }}\\
		\vspace*{2cm}
		{\LARGE\textsc \@title }\\
		\vspace{0.2cm}
		{\LARGE \textit \@subtitle}\\
		\vspace*{1cm}
		{\large par\\
			\vspace*{0.1cm}
			{\bfseries\@author}}\\
		\vspace*{0.8cm}
		{\large Thèse de doctorat de \@rfield}\\
		\vspace*{0.5cm}
		{\large dirigée par\\
			\vspace*{0.1cm}
			\@supervisors}\\
		\vspace{1cm}
		Présentée et soutenue publiquement le \@defdate\\
		\vspace*{1cm}
		Devant un jury composé de : \\
		%[Noms, Prénoms, fonctions (ex : Maître de conférences) et rôle (ex : Rapporteur) des membres du jury]
		\vspace{0.5cm}
		
		\jury
		
		\vfill
	\end{center}
\end{titlepage}

\restoregeometry
	\makeatother %makes @ as a symbol/number again (comes with \makeatletter in the preamble)
	
	%%% Dedication page %%%	
	%\afterpage{\blankpage} %blank page
	\cleardoublepage 
	\thispagestyle{empty} %deletes header and footers of the page
	\vspace*{5cm}
	\null\hfill\textit{[Dedicace]}

	\pagenumbering{roman} % numbering of each page with roman numerals (i,ii,iii...)$
	
	\cleardoublepage
	
	%%% Acknowledgments/Remerciements %%%
	\chapter*{Acknowledgments/Remerciements} 
	\thispagestyle{plain} %deletes header of the page
	\lipsum[1-3]
	\cleardoublepage
	
	%%% Tables of contents/Sommaires %%%
	\dominitoc %to make small table of contents/sommaire at the beginning of each chapter [can be deleted if not needed]
	\tableofcontents %print the global table of contents/sommaire of the manuscript
	\cleardoublepage

	\pagenumbering{arabic}
	
	\chapter*{Introduction}
	\addcontentsline{toc}{chapter}{Introduction} %adds this section in the table of contents	
	\markboth{Introduction}{Introduction} %header
	\lipsum[1-5]
	\thispagestyle{plain} %deletes header of the page
	
	\adjustmtc %to adjust minitoc after a unumbered chapter (like introduction before)
	
	\chapter{Title of the chapter 1}
	\minitoc % Creates the toc of chapter 1
	\section{Section 1.1}
	\subsection{Subsection 1.1}
	\subsubsection{Subsection 1.1}
	\paragraph{Paragraph 1}
	\lipsum[1-15]
	
	\section{Section 1.2}
	\subsection{Subsection 1.2}
	\subsubsection{Subsection 1.2}
	\paragraph{Paragraph 1.2}
	
	\chapter{Title of the chapter 2}
	\minitoc % Creates the toc of chapter 2
	\section{Section 2.1}
	\subsection{Subsection 2.1}
	\subsubsection{Subsection 2.1}
	\paragraph{Paragraph 1}
	\section{Section 2.2}
	\subsection{Subsection 2.2}
	\subsubsection{Subsection 2.2}
	\paragraph{Paragraph 2.2}
	Normal size here
	

	%%% Print the table of figures %%%
	\addcontentsline{toc}{chapter}{Figures} %adds this section in the table of contents
	\listoffigures 
	\cleardoublepage
	
	%%% Print the table of tables %%%
	\addcontentsline{toc}{chapter}{Tables} %adds this section in the table of contents
	\listoftables 
	
	
	%%% Bibliography %%%
	%\renewcommand\bibname{References} %changes the name of the bibliography
	\addcontentsline{toc}{chapter}{\bibname}
	\bibliographystyle{unsrt}
	\bibliography{references}
	
	%%% Appendix %%%
	\appendix %starts numbering chapter/sections with A (e.g. A.1, A.2...)
	\chapter{name of appendix 1}
	\chapter{name of appendix 2}
	\lipsum[1-2]
	
	%\newpage
	\cleardoublepage
	\thispagestyle{empty}
		
	
	%%% COVER PAGE RESUME(fr)/ABSTRACT(en) %%%
	\cleartoleftpage %to make this page on the left side
	\newgeometry{left=2.5cm,right=2.5cm,bottom=0.1cm,top=3cm} %centers the page with equally margins
	\thispagestyle{plain}
	
	%% the following commands add a frame around the résumé/abstract
	\tikz[overlay, remember picture] \draw ([xshift=1cm,yshift=-1cm]current page.north west) rectangle ([xshift=-1cm,yshift=1cm]current page.south east);
	\tikz[overlay, remember picture] \draw ([xshift=.7cm,yshift=-.7cm]current page.north west) rectangle ([xshift=-.7cm,yshift=.7cm]current page.south east);
	%--%
	
	\section*{Résumé}
	\lipsum[1-2]\\
	%replace lipsum[1-2] with your the résumé here (in french)
	\noindent Mots clés : [----------]\\ %add mots clés
	\vspace*{1cm}
	
	\noindent \textbf{[Titre de la thèse en anglais]}
	\section*{Abstract}
	\lipsum[1-2]\\
	%replace lipsum[1-2] with you abstract here in english
	
	\noindent Keywords: [----------] %add keywords
		
\end{document}

